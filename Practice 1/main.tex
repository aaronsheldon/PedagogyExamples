\documentclass{exam}
\usepackage{longtable}

% Bibliographic elements
\title{DATA3304 Practice 1}
\date{\today}
\author{Bow Valley College\\Data Management and Analytics}

% Header
\pagestyle{headandfoot}
\firstpageheader{DATA3304}{Practice 1}{February 9, 2024}
\firstpageheadrule
\runningheader{DATA3304}{Practice 1}{February 9, 2024}
\runningheadrule

\begin{document}
Name:\enspace\hrulefill
\begin{questions}

    \question For each of the two scenarios identify whether the rows of the example data
    are a statistically independent sample, or if the rows are dependent on each other.
    Please provide a one or two sentence explanation.
    \begin{parts}
        \part A web-store has collected sales data of unrelated clients, who are
        geographically dispersed. Each record contains the total sales made to a single
        client in the previous fiscal year. Empty registration dates indicate the client
        created an account before the start of the previous fiscal year.
        \begin{longtable}{lllllll}
            Client & Sales & Sessions & Shipments & City & Registration & $\cdots$\\
            \hline
            \endhead
            A & \$500 & 25 & 2 & X & June 5 & $\cdots$\\
            B & \$2000 & 50 & 5 & Y &  & $\cdots$\\
            C & \$250 & 12 & 1 & Z & October 23 & $\cdots$\\
            D & \$1200 & 61 & 4 & W &  & $\cdots$\\
            \vdots & \vdots & \vdots & \vdots & \vdots & \vdots & $\ddots$
        \end{longtable}
        \vspace{\stretch{1}}
        \part A social-media application has collected the time evolution of the
        self-reported relationships between pairs of users. Each relationship is represented
        by a pair of records, each containing a pair of users, the type of relationship, and
        the time-span of the relationship.
        \begin{longtable}{llllll}
            Request User & Relation & Respond User & Start & End & $\cdots$\\
            \hline
            A & Daughter & B & & & $\cdots$\\
            B & Mother & A  & & & $\cdots$\\
            A & Friend & C & March 5 & & $\cdots$\\
            C & Friend & A & March 5 & & $\cdots$\\
            D & Employee & B & May 23 & December 15 & $\cdots$\\
            B & Employer & D & May 23 & December 15 & $\cdots$\\
            \vdots & \vdots & \vdots & \vdots & \vdots & $\ddots$
            \endhead
        \end{longtable}
        \vspace{\stretch{1}}
    \end{parts}

    \newpage
    \question A taxi company has collected shift length and number of ride statistics over
    a fixed time period. Unfortunately the data report has overstated the precision of
    measurement in an effort to improve data quality, using partial fractional attribution
    for records overlapping with the boundaries of the time period.

    The report stated that there were 2766.45 minutes of driver-shift recorded, but the
    source data system only records shift lengths in increments of 10 minutes.
    
    Furthermore, the report stated there were 212.85 rides. We know that it is only possible
    to measure ride counts in increments of 1.
    \begin{parts}
        \part Using rounding at half away from zero please round each measurement to the
        reliable digits, and provide the marginal uncertainty.
        \vspace{\stretch{1}}
        \part Assuming rides are requested randomly and independently from the drivers, find
        the marginal uncertainty of the ratio of rides over driver-shift length, and provide
        the units of measurement of the ratio through dimensional analysis. Please show your
        work.
        \vspace{\stretch{1}}
    \end{parts}

    \newpage
    \question It has been purposed to implement random COVID screening of students using a
    rapid test. Currently 1 in 1000 Albertans have COVID. Please use the outlined Rapid Test
    Confusion Matrix in your answer.
    \begin{longtable}{lp{3cm}p{3cm}}
        & \multicolumn{2}{c}{Actual COVID Infection} \\
        Test Result & Absent & Present\\
        \hline
        \endhead\\
        Positive & $1/100$ \par False Positive & $2/3$ \par True Positive\\\\
        Negative & $99/100$ \par True Negative & $1/3$ \par False Negative
    \end{longtable}
    \begin{parts}
        \part If a student tests positive what is the incorrect conclusion reached by the
        Prosecutor's Fallacy, please specify either of the incorrect probabilities from the
        confusion matrix.
        \vspace{\stretch{1}}
        \part What is the correct conclusion and probability of being infected, please show
        your work.
        \vspace{\stretch{1}}
    \end{parts}
\end{questions}
\end{document}